\chapter{Conceptos Básicos}


\section{Reconocimiento del Habla en Humanos\cite{gales}}

Si bien el acto de comunicación empleando el habla se lleva a cabo de manera natural, y sin esfuerzos aparentes, este proceso es altamente complejo, como eficiente. Esa eficiencia se expresa en su robustez ante diferencias de voces,
hábitos o estilos de habla, dialectos y acentos de locutores, ruidos, distorsiones e interferencias. 

Se puede ver la comunicación humana como una cadena de eventos que vinculan el cerebro del locutor y del oyente, y que se conoce como \textbf{cadena del habla} (figura \ref{fig:cadenahabla}).
\begin{figure}[H]
\centering
\includegraphics[scale = 0.45]{figure/CadenaHabla.jpg}
\caption{Cadena del Habla}
\label{fig:cadenahabla}
\end{figure}


Considere la interacción entre dos personas, donde el locutor quiere transmitir información al oyente empleando el habla. Lo primero que debe hacer es ordenar sus pensamientos, establecer qué es lo que quiere decir y transformar esa idea en un arreglo de unidades lingüísticas a través de la selección de secuencias de palabras y frases.
Tal ordenamiento debe seguir restricciones gramaticales del lenguaje usado. Todo este proceso de imaginación, selección y ordenamiento de unidades de acuerdo a normas del lenguaje se lleva a cabo en el cerebro del locutor casi instantáneamente. A este nivel se lo puede llamar \textbf{nivel lingüístico} de la cadena de habla. 

También a nivel del cerebro del locutor se codifica el mensaje planificado en instrucciones motoras. A través de los impulsos eléctricos correspondientes a esas instrucciones motoras, los nervios activan y coordinan la actividad muscular de los órganos vocales: los pulmones, las cuerdas vocales, la lengua, las mandíbulas, los labios.

Este es el \textbf{nivel fisiológico}, ya que involucra a los eventos de actividad neuronal y muscular del locutor.

Finalmente, el efecto de la activación del aparato fonador provoca patrones de variación en la presión del aire, denominadas ondas de sonido, que se propagan hasta los oídos del oyente, y del propio locutor. Al hablar el locutor oye lo que articula, y emplea esa información para retroalimentar su mecanismo de fonación, comparando la calidad de los sonidos que intentó producir con los que realmente produjo y
efectuando ajustes necesarios para minimizar las diferencias.
Este proceso de generación y transmisión de las ondas acústicas se corresponden con el nivel \textbf{físico o acústico}.

Del lado del oyente, los patrones de presión de aire que fueron provocados por la fonación del locutor, provocan movimientos en la membrana timpánica, que se transfieren a través de un sistema mecánico ubicado en el oído medio hasta el oído interno. Ya en el oído interno esos patrones de movimientos se transducen a impulsos eléctricos, que
viajan a lo largo de los nervios acústicos hasta el cerebro. La llegada de los impulsos nerviosos modifica la actividad nerviosa que se estaba produciendo en el cerebro del oyente, y a través de un proceso que aún no se conoce completamente, se reconoce el mensaje del locutor. Es decir que del lado del oyente, el proceso se revierte: los eventos
comienzan en el nivel físico, donde las ondas acústicas activan el mecanismo auditivo, continúan en el nivel fisiológico, donde se procesa la información auditiva, y termina en el nivel lingüístico en el momento que el oyente reconoce las palabras y frases pronunciadas por el locutor.

Cabe aclarar que el modelo presentado es claramente una simplificación de lo que en realidad ocurre. Por ejemplo, no aparecen los estímulos visuales que también se emplean simultáneamente con los auditivos para reconocer el habla.

\subsection{ Organización Lingüística del Habla}

El mensaje antes de ser transmitido al oyente es estructurado lingüísticamente por el locutor. A través de este proceso el locutor selecciona las palabras y oraciones adecuadas para expresar un mensaje.

Se puede considerar que el lenguaje está conformado por una serie de unidades. Estas unidades son símbolos que permiten representar objetos, conceptos o ideas. El lenguaje es un sistema conformado por esos símbolos y reglas para combinarlos en secuencias que permitan
expresar pensamientos, intenciones o experiencias, y que aprendemos a identificar y utilizar desde niños.

Se pueden definir las siguientes unidades básicas, para el estudio de los distintos niveles lingüísticos:\\




\begin{defn}
 Un \textbf{Fonema} fonema es la unidad mínima de diferenciación del lenguaje y hace referencia a un sonido contrastivo en una lengua determinada. 
\end{defn}

\begin{itemize}
\item Cada lengua tiene un número reducido de fonemas. En el Español existen cinco sonidos vocálicos y menos de veinte sonidos consonánticos (dependiendo del dialecto).
\item Aunque en sí mismo un fonema no tiene significado alguno, al reemplazar un fonema por otro en una palabra podemos obtener una palabra diferente.
\item El conjunto de todos los sonidos (fonos) que son aceptados como variantes de un mismo fonema se conoce como \textbf{alófonos}. Por ejemplo, si se analiza la forma en que se pronuncia la palabra “dedo” de manera aislada, se puede encontrar que sus dos consonantes, a pesar de corresponder al mismo fonema: “d”, se
pronuncian de manera diferente. Mientras la primera se realiza apoyando el ápice de la lengua contra los dientes superiores, de manera tal que se impide el paso del aire (consonante oclusiva), la segunda se pronuncia sin llegar a provocar una oclusión total (articulación aproximante). Estos dos sonidos son variantes
alofónicas del fonema “d”.
\item Al estudiar la estructura fónica de una lengua se debe considerar tanto los fonemas como sus principales alófonos. Debido a que la pronunciación de todos los sonidos varía dependiendo de aspectos como la influencia de sonidos cercanos, la rapidez de elocución, o el estilo de habla, se puede encontrar un considerable margen de detalle con el que se puede caracterizar a los
sonidos. Por ello los lingüistas suelen hacer una distinción entre representación fonética estrecha, en las que se incluye un gran número de detalles de pronunciación, y representación fonética amplia en donde se incluyen los detalles no contrastivos considerados más importantes.
\item Los hablantes de un idioma tienden a oír solamente las diferencias entre sonidos que son relevantes para distinguir una palabra de otra para esa lengua. Es decir que muchas veces no se distinguen entre sí las variantes alofónicas. Por otro lado, diferencias alofónicas para una lengua pueden ser fonémicas en otra, por ejemplo “pelo” y “pero” pueden sonar indistintos para el Japonés
o Coreano, mientras que para el Español presentan diferencias fonéticas.
\end{itemize}
Al clasificar los sonidos del habla suele hacerse una distinción entre sonidos consonánticos y vocales. En la articulación de las consonantes se produce una obstaculización u obstrucción al paso de aire procedente de los pulmones. Durante la producción de vocales, en cambio, el aire pasa por la cavidad bucal sin tales
obstáculos.

Los sonidos de las vocales se pueden clasificar utilizando tres parámetros, dos que tienen que ver con la posición de la lengua: su \textbf{altura} y su \textbf{desplazamiento} hacia la parte anterior o posterior de la boca, y el tercero vinculado con la \textbf{posición de los labios}.
\begin{itemize}
\item Teniendo en \textbf{cuenta la altura del dorso de la lengua} se distinguen vocales altas, (/i/ y /u/), vocales medias (/e/ y /o/); y vocales bajas, con descenso del dorso, que en Español es únicamente la /a/.
\item Según el desplazamiento de la \textbf{lengua hacia adelante o hacia el velo}, tenemos vocales anteriores (/i/ y /e/), una vocal central (/a/), y vocales posteriores, con retracción del dorso (/o/ y /u/).
\item Según  la \textbf{disposición de los labios}, tenemos
dos vocales redondeadas (/o/ y /u/) y tres no redondeadas (/i/, /e/ y /a/).
\end{itemize}
Los sonidos consonánticos por su parte se suelen clasifican según tres parámetros principales: punto de articulación, modo de
articulación y actividad de las cuerdas vocales.

\begin{itemize}
\item El \textbf{punto de articulación} especifica cuáles son los órganos articulatorios que provocan la oclusión o impedimento a la circulación
del aire expelido. En este caso al articulador en movimiento se
denomina activo y al que permanece inmóvil o presenta menor
movimiento articulador pasivo. Por ejemplo, la “p” es un consonante bilabial, la “t” es ápicodental, y la ”k“ es velar.

\item El \textbf{modo de articulación} describe la naturaleza del obstáculo que se encuentra a la salida del aire. De acuerdo a esta clasificación se puede separar a las consonantes en oclusivas, fricativas, africadas, aproximantes, nasales, laterales y vibrantes.
\item De acuerdo a la \textbf{actividad de las cuerdas vocales} se puede diferenciar a los sonidos sordos, producidos sin la vibración de las
cuerdas vocales, y los sonidos sonoros, articulados con vibración
de las cuerdas vocales.\\
\end{itemize}
  

\begin{defn}
Un \textbf{Morfema} es la unidad mínima de significación del lenguaje.
\end{defn}

\begin{itemize}
\item Hay dos tipos de morfema: el \textbf{lexema}, que en un contexto más
coloquial se conoce como “raíz de una palabra” y es su principal
aporte de significado, y el \textbf{gramema}, que es un morfema generalmente pospuesto al lexema para indicar accidentes gramaticales,
y pueden ser de género o número.
\item También hay gramemas independientes como las preposiciones.\\

\end{itemize}

\begin{defn}
Lexía son unidades léxicas compuestas por morfemas relacionados mediante un alto índice de inseparabilidad, o un agrupamiento es de semas, que constituyen una unidad funcional.

\end{defn}



